\documentclass[fleqn,10pt]{olplainarticle}
% Use option lineno for line numbers 

\title{Measuring Commuting Patterns of Under-represented Populations Using Digital Trace Data}

\author[1]{Neal Marquez}
\author[1]{Sara Curran}
\author[2]{Adrian Dobra}
\affil[1]{University of Washington, Department of Sociology}
\affil[2]{University of Washington, Department of Statistics}
\keywords{Digital Trace Data, Commute Patterns, Immigrant Populations}

\begin{abstract}
Traditional statistics on home-to-work commutes often come from one of two data sources: the American Community Survey (ACS) and the LEHD Origin-Destination Employment Statistics (LODES). While both are frequently used, they fail to capture the commute burden that is experienced by individuals who work in quasi formal or informal labor sectors such as construction, domestic service, agriculture, or 'gig' economy work where work sites or workers may not be recorded or work sites are not consistently observed. Historically, work in these sectors have been over-represented by low-income, foreign-born, Black, and Hispanic populations and, as such, studies which rely on LODES and ACS may miss an important aspect of the commuting patterns of these populations. To better capture work-arrangement commuting across a full array of workers, we utilize a novel digital trace data source, the SafeGraph Mobility Data Set. Trace data from mobile apps is growing in use for uncovering mobility not otherwise observed or to supplement other sources of mobility data. In this case, we evaluate the extent to which SafeGraph data might complement or supplement existing data sources for understanding mobility. We find that SafeGraph data is more similar in its geographic representation of commuters to ACS population estimates of low-income and predominately Black and Hispanic neighborhoods and observed mobility patterns significantly differ from LODES.

\end{abstract}

\begin{document}

\flushbottom
\maketitle
\thispagestyle{empty}

\section*{Introduction}

National comparative statistics on home-to-work commutes have often been used to compare the mobility patterns of populations between income brackets \citep{}, geography \citep{}, race and ethnicity \citep{}. While these studies have found significant differences in patterns of mobility between these groups, the data sources most often used for these analysis may not fully describe the patterns of mobility these groups undertake. Biases in who is monitored as well as what mobility patterns of an individual are identified are limited by the data collection methods of traditional commuting data sources \citep{}. This may be especially problematic for individuals participating in informal labor sectors such as construction, domestic service, agriculture, or 'gig' economy which have been historically over represented by low-income, foreign-born, Black, and Hispanic populations \citep{}.

This study looks to circumvent limitations of observing the mobility patterns of traditionally missed populations by using an alternative data source, the SafeGraph Mobility data set. Using digital trace data in the state of Washington from SafeGraph Inc. we calculate measures of home to most visited location flow analogous to measures in more traditional geographic mobility data. We find that SafeGraph data is more similar in its geographic representation of commuters to ACS population estimates of low-income and predominately Black and Hispanic neighborhoods and observed mobility patterns significantly differ from LODES.

\section*{Motivation}

Statistics on commuting flows within metropolitan areas are an important building block for city planning, especially for transportation-related projects \citep{McKenzie2013}. Understanding how individuals move within a city is vital to provide transportation services and other public amenities that best support the population. Though locations are more frequented when access to them is made easier by transportation services, bias exists in whose preferred locations have best transportation access. For example, previous studies have found that cities are often constructed in such a way that benefit more affluent populations over impoverished ones in their transportation needs \citep{soja_seeking_2010, Williams2014}.

Among the number of reasons that transportation infrastructure may not well suit the needs of certain segments of the population is the way in which transportation flow data is collected. Transportation flow data in the United States most often comes from linked administrative data or survey data and is reported by the Census Bureau. The Census Bureau's predominant focus is on work-home flows and work-home commuting flows are often the evidence base used for changes in transportation projects \citep{McKenzie2013}. Work-home transportation flow statistics are produced from either the American Community Survey (ACS) or the Longitudinal Employer-Household Dynamics Origin Destination Employment Statistics (LODES), however noticeable differences are found between the two. Design differences between the the two sources lead to different narratives on commuting behavior because of the populations that are captured \citep{Graham2014}.

The ACS is an annual household survey that was intended to be a more cost efficient and regular manifestation of the census long form. Aside from small changes to the questions that had been asked, the primary difference between the census long from and the ACS is the sample size. While the long form was a 1 in 6 sample the ACS is a 1 in 50 sample of households in the United States. Because of this ACS data is often aggregated over five year periods for many estimates, including commuter work flow data, to reduce the sampling error \citep{Green2017}. Because of the household focused nature of the survey the ACS capture a majority of the population, save for those without a traditional residence, regardless of their employment, citizenship status, or job sector. The limitation of the ACS however comes from the limited number of questions that are asked regarding commutes to work. Commuter data is derived from the location worked at most in the week prior to the date of the survey. This means that each survey respondent may only provide one place of work, despite how many locations that job may require them to go or additional jobs that a respondent may have. In addition, data is reported at the county level which does not well capture the needs of more local populations such as neighborhoods.

The lack of detail on the number and location requirements for jobs has a significant impact on the underestimation of commute time and distance for those individuals with multiple jobs. In particular, for low income immigrant populations who often work more than one job \citep{Menjivar1999a} and who have jobs without a regular location \citep{Chaufan2012}, the ACS does not well characterize the extent to which low income immigrant populations may experience what some authors refer to as transport disadvantage \citep{Delbosc2011}. 

LODES data, on the other hand, reports on all commuter flows by linking residence to place of work based on Federal and State administrative data and captures more than 130 million flows annually \citep{}. Flows are measured at a much more granular level than the ACS, block rather than county, and are dis-aggregated by age of individual, earnings, and job sector. Furthermore, any job which is covered under unemployment insurance is tracked by this process and included in the final published data set \citep{}. This process better captures individuals who work multiple jobs and, to some degree, jobs with multiple locations, however, the method is not without fault and may systematically miss populations which participate in particular labor sectors. First, commutes are not directly calculated from individual travel routes but instead by taking the reported locations of the place of employment and the place of residence of the employee. This measure of "flow" is not a directly observed flow but rather an implied flow which may mischaracterize the true destination of an individual. Though Employers are asked to report all places of work, this is not always done or is simply not possible. For example, some employment sectors with irregular places of work, such as agricultural, landscape, and construction jobs require individuals to report to locations that are not associated with the company or a regular place of work \citep{}. In these cases flows are misrepresented by having the wrong destination which may miss out on important movements made by individuals between neighborhoods. Second, individuals who are self-employed or act as independent contractors are not covered by unemployment insurance and thus not captured in the LODES data. Again, for labor sectors such as landscaping and construction where independent contractor employment is more common, individuals participating in such labor sectors are more likely to be systematically missed in the LODES dataset.

For those labor sectors which are most systematically missed by the LODES data set, general demographic characteristics of the individuals participating in those labor sectors are more representative of particular groups. For example, agricultural, landscape, and construction jobs tend to be occupied by individuals in lower income brackets \citep{} and jobs are disproportionately filled by Hispanic individuals. Foreign born Hispanics in particular, are heavily over-represented in the construction industry \citep{}.

When assessing the needs of a cities population in terms of their commuting locations and times, traditional data sources do not well describe individuals participating in agricultural, landscape, and construction jobs which are over-represented by low income, immigrant populations. This lack of representation in either data set has implications for future city planning that leaves out the low income immigrant population and underestimates the current transportation burden experienced some populations \citep{Choi2013}. Rather than using survey or administrative data to estimate work flows it may be of more use to estimate commute flows for populations not well represented from other data sources using digital trace data. Digital trace data has been used in a number of migration studies \citep{Hughes2016} and reports have found that migrant populations rely heavily on mobile phone usage in order to access resources and maintain their social networks \citep{Maitland2018}. In addition, using digital trace data such allows us to how more granular spatial-temporal measures of positioning to calculate commuter flows rather than infer them from time agnostic data sources such as the ACS and LODES data sets.

\section*{Data and Methods}

Our primary data source is an app-based digital trace data source from the SafeGraph group. Data comes from cellphone device pings in the state of Washington which collect time stamp information, GPS location, and accuracy of the GPS location in meters and are associated with an anonymized device id. Pings occur at irregular intervals and data collection is, in part, dependent on the ways in which a user interacts with their device. In total 2,566,681,894 pings were collected from 6,974,314 devices from the period of October 31st, 2018 to January 31st 2019 within the state of Washington. [Neal, you could state here that SafeGraph data is available worldwide, and that SafeGraph is only one of several companies that are currently offering data on human mobility generated from unsolicited mobile phone locations. This is important to say since it demonstrates why looking at these data is essential. You could say that these data are collected and merged in databases of locations in real time.]

SafeGraph also provides estimated home geographies of individuals at the block group level. Home locations are derived from the total observed patterns of the individuals ping behavior which is not limited to either Washington or the three month time span of our own data. We limit our analysis to only those individuals who have an estimated home location within the state of Washington. This filtering process leaves us with 658,531,919 observations from 438,976 devices.

To assess the temporal distribution of the SafeGraph data we visually inspect the day of week trends of observed data in five minute intervals across the state of Washington. Figure \ref{} shows the average user count of observations by day of week and 10 minute interval. The graph gives us an idea if their is a bias in when data is collected and what types of movements we may be missing. Data follows a somewhat expected pattern with a peak number of users being observed between 4:00 and 5:00 PM PST on all days and low counts of users being observed in the early morning hours of 3:00 and 5:00 AM PST. Saturday and Sunday appear to have slightly different patterns of average numbers of observed users which may be due to different phone use behavior patterns \cite{}.

We also assess the daily pattern of observed users across the three months of total observed individuals. A general decline over the three month period can be observed in figure \ref{}. In addition, daily patterns show sizeable declines in the observed number of users during the holiday period between the 20th of December 2018 and the 2nd of January.

Several steps are taken to make SafeGraph data comparable to more traditional data sources of commuting patterns. First each ping is geotagged to the census tract from which the data point originated using the GPS coordinates. By doing so, we lose precision of the SafeGraph data in exchange for being able to associate points with U.S. census tracts (2018 boundary definitions). Second, we calculate time spent in an area. For each individual, we order observed points in time and group together points that are consecutive and originate from the same census tract. For each group of points in the same census tract we treat the total amount of time spent in that tract as the time difference between the last and first point observed in the group. Thus, if an individual passes into a census tract and only a single ping is recorded in that tract, we assign no time spent in that tract even though the individual was observed in that location at a point in time (see figure \ref{} for more details). We assume that individuals with consecutive pings within a census tract are in the same location which may be violated if an individual leaves a census tract and then later returns to that tract with no pings recorded during that time. For each individual, we aggregate the total amount of time spent in each census tract and divide by the total amount of time observed all census tracts to get the proportion of observe time spent in each tract.

We are most concerned with tracts which have a large amount of time spent in its boundaries by the user. We order tracts for each user by total time spent in the census tract from least to greatest and record the cumulative amount of time spent in all tracts. For our analysis we consider all tracts which fall within the first 95\% of the sorted cumulative time spent in tracts as locations which a device frequently visits. Figure \ref{} shows the average cumulative distribution of time spent in the nth most visited locations with individuals having an average of 2.1 locations in the first 95\% of their cumulative distribution of locations visited. These locations are treated as "destination" locations while we treat the census tract of the home that is provided by SafeGraph as the "origin" location, which together account for a "flow". Note that we do not observe the direct flow from the origin to the destination but rather simply account for a destination as the location one spends a significant portion of their observed time with origin being fixed at the home location. This definition of origin and destination makes our data set more comparable to traditional data sets of commuting statistics. We additionally remove all flows that have a destination which matches the home location as we can not distinguish time spent at the origin vs a destination within the home census tract and because we are most interested in flows between neighborhoods. In total, we observe 311,278 flows from 134,196 individuals.

LODES data is used as a our primary source data for comparison. We treat each origin destination pair as a unique "flow". While flows are reported at the block level with 2010 definitions of block geographies we aggregate data to the tract level. Only flows within the state of Washington are used in the analysis and we remove flows which originate and end in the same census tract giving us a total of 2,937,567 flows.

In order to address concerns of representation among a number of demographic dimensions we compute measures of entropy from both SafeGraph and LODES to Census population estimates taken from the 2018 American Community Survey (ACS). We use ACS estimates of population as a "ground-truth" to compare to our observed population of individuals and flows in SafeGraph and LODES. First, we compare the distribution of origin census tracts from SafeGraph and LODES to ACS estimates of total population. From the ACS we calculate $p_i$ as the estimate of the population within census tract $i$ divided by the total state population. We than calculate $q_{i,s}$ as the observed number of flows originating from census tract $i$ from source $s$ divided by the total number of flows from sources $s$, where $s$ may be either SafeGraph or LODES. From here we are left with $\boldsymbol{p}$, $\boldsymbol{q_\text{LODES}}$, and $\boldsymbol{q_\text{SafeGraph}}$ which are vectors or relative population distribution across the 1,458 census tracts in Washington. We calculate entropy between either source and ACS estimates by using Kullback–Leibler divergence (KLD). KLD measures how one probability distribution is different from a second, reference probability distribution \citep{Kullback1951} with a measure of 0 indicating that the two distributions are identical. We use ACS estimates as the reference distribution and the subsequent KLD measure as a metric for which distribution is more similar to the "ground-truth". Though one source may be more similar to the ACS by geographic distribution we should also be concerned about who is represented in those geographies. To account for this we use the estimate of Median Household Income for each tract from the ACS and assign an income quintile based on this value. Populations are then aggregated and normalized such that $r_i$ is the number of individuals who live in census tracts falling in the $i\text{th}$ quintile divided by the total population of Washington state. We again calculate comparable measures from SafeGraph and LODES for which KLD is calculated again using the ACS estimate as the reference. Because we are also concerned with measures of racial and ethnic representation we also assign a majority racial or ethnic group to each tract by examining the total population of each tract who is Non-Hispanic White, Non-Hispanic Black, Non-Hispanic Asian, or Hispanic and assigning a group to a tract based on the largest group population. Again, comparable measures are calculated for SafeGraph and LODES and a measure of KLD is reported.

The ACS provides a baseline measure to which we may compare both data sets with respect to who is represented in the population but we are concerned with how SafeGraph and LODES differ in their measures of flow. Using the assignments of income quintile we estimate $P_s(z|y)$ which is the probability of observing a flow from a census tract with income quintile $y$ to a census tract with income quintile $z$ from source $s$. In order to quantify uncertainty of these measures we bootstrap our statistic by sampling flows from the SafeGraph and LODES sources independently and compare the results. A similar process is repeated using racial and ethnic group assignments for census tracts as well.


\section*{Results}


\newpage
\bibliography{WorkPlaceTravel}

\end{document}